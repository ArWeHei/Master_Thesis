\setchapterpreamble[u]{\margintoc}
\chapter{Introduction}
%\labch{intro}

Since the arrival of electric-based computers in the 20th century, the range of 
application for computers and algorithms has steadily grown.
From the first simple arithmetic calculations over solving complex equations and simulating
realistic systems to media production and funny apps for phones; computers seem to
seize every aspect of everyone's life.
This includes areas that stood their ground for long time against the automation
that has become common for so many areas like production, communication \etc.
Recent advances in Machine Learning and the ever increasing efficiency, availability
and sheer power of computers has put the thought of professions like lawyers and physicians being
'future-proof' into question.

This includes even the field of arts.
As art comes can take pretty much every form there is and as it depends on the viewer,
it is valid to ask whether art could ever be understood by anything but a human being.
Nonetheless, people have tried repeatedly and proven that Machines CAN grasp broad
concepts in art, even the delicate art of drawing

\paragraph{Evolution of Art and Style Transfer}
Since the late 1980's the field of computer vision has looked not only at photographs
and videos but also started working on images of artworks.
From early computer based analysis and identification of images that is used even
today for identifying counterfeits, the field has gone as far as creating art on
its own.
Ever since Leon Gatys introduced his paper on 'Neural Style Transfer' in 2016, the quality
of computer generated paintings has come closer and closer to being indistinguishable to
an image of an original artwork.
%At the same time techniques have continuously improved in order to replicate the style of
%images, periods, techniques or artists.
This resulted in many interpretations of given images in the style of well-known
artists like van Gogh, Matisse or Picasso.

\paragraph{Representation Gap}
Yet, many of these accomplishments lack the final step towards creating authentic artworks,
which is the transitions from the image domain to the actual artwork domain.
As images are mere projections of the artworks they try to represent, they lack 
some of the original artwork's content and message.
Most people would agree that it is an entirely different experience to view a painting
in real life than viewing an image of this painting on a screen or sheet of paper.

For once, paint on canvas has depth to it, that can hardly be visualized by a 2D plane.
Every viewing angle of an image gives away different details in the works of the artist.
How much paint did the artist use?
How many layers of color were applied?
Which brush or technique did were used?
All of this plays into an artwork's impression as much as the distribution of color does.
This is the reason why people are interested in actual painted replicas of their favorite
artworks, instead of just high resolution prints.

As this stands, it becomes clear why an image will always loose some of this information
about the artwork.
Sure, one could always make hundreds of very high resolution images or even scan
the image in 3D space, yet without advanced technology like VR headsets or the like,
it still does not quite replace the real deal.

\paragraph{From Pixels to Strokes}
So why even bother then?
Even with all the shortcomings which were described here, images still catch most of
the essence of an artwork -- at least enough to train neural networks on this.
But is it possible to maybe depict artworks as what they actually are, a composition
of color particles on canvas?
It just so happens that there are works such as \cite{adobe stroke paper} that
introduced the idea of synthesizing brush strokes through fluid dynamics and neural networks.
Unfortunately this comes with a high computational burden.
So maybe one could start by at least taking a step towards this sort of representation
by depicting an artwork through a set of brush strokes.
Ideally, this could help to better capture the essence of an artwork is this comes
closer to its real world representation and later even open the door for style transfer
to be performed on this representation as well.


\paragraph{Structure of this work}
In this work an attempt at retrieving such representations for artworks is made.
First, the theoretical background for such an attempt along with related work will be presented.
Then the approach itself will be inferred in two parts that describe the training of a differentiable renderer
and the subsequent retrieval of brush strokes from an image.
This is followed by experiments as well as an ablation study of the approach.
Lastly, there will be a discussion about the results and the future direction of research.



\paragraph{Allocation of individual contributions}

