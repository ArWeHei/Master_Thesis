%\setchapterpreamble[u]{\margintoc}
\chapter{Introduction}
\labch{intro}

% difference to 'drawing' -> different media
% shortcoming of computers to visualize this -> will draw enormous resources
% paintings are most often displayed though images -> no texture

\todo{this need at least some pictures in it}

The American Cambridge dictionary defines the word "picture" as
    
\dictcite{CAD}{picture}
\todo{add phonetic signs}
 
which right away separates between different media that can make up a picture.
If one were to look up the definitions of "drawing", 'painting" and "photograph" as well, this is what distinguishes the three words from one another.

Paintings are only pictures that are made with paint.

A drawing confines oneself to pictures drawn with a pencil or pen.

And finally, a photograph is a picture taken with a camera.

All of these three methods could be used to picture the same image, but will always give different results, as each medium comes with its own limitations.

The painting will almost always deviate in its details from the original image, as the paint tends to mix and flow.
The drawing will present large evenly colored areas often in a grainy way due to the structure of the paper which it has been drawn on.
The photograph will limit an artist in showing anything more than the raw visual input.

Yet, all of these techniques also have their own characteristics that other techniques do not possess.

When comparing paintings to photographs this becomes especially clear.
The layered texture of paintings which varies vastly in thickness within a single painting is something that photographs will never be able to capture and display, as photographs are just the projection of a 3D world onto a 2D plane.
This will not only be confirmed by art historians or the like, but can be seen when looking at the industry of painting replicas of masterpieces with oil such as the Starry Night by van Gogh or most famously the Mona Lisa by DaVinci.
There can only be a market for such replicas can only if there is enough of a difference between an actual painting and a very high resolution photo of a painting.

This does not hold as well for drawings on the other hand. As drawings are also inherently a 2D representation with just an infinitesimal height to them, they CAN be captured by photos quite well.
Also, as technology has become more and more advanced, it is possible to capture and print a drawing in such a high quality and fidelity to the original drawing, that it is hard to tell the original drawing and the print apart.

This shows also in the popularity of drawing applications on computers and tablets that have been useful tools to digital artists for over 2 decades now.
And further development it has now become a feasible challenge to mimic the feel of pen on paper with styluses on glass tablets.

All of this makes it clear, that the art of drawing has been object to the digital revolution as much as the art of taking photographs.

But what about paintings?
Are there not also applications that try to imitate painting techniques, or 3D printed replicas of famous paintings?
Yes there are, but it is clear that this comes with a lot of limitations or huge effort, as 3D scans and prints and real-time fluid simulations are cutting-edge technology.
And even then the majority of digital content is consumed though 2D screens which are not able to display what makes a painting unique.

So why even bother then, if all of these limitations are in place?

As it already has been hinted, there is existing technology that is capable of taking paintings into the digital realm.
At the same time Augmented Reality (AR) and Virtual Reality (VR) gain more traction every year, which will only strengthen this development.

Thus, the question arises: Is it possible to digitalize paintings?
One way to achieve this are the mentioned 3D scans and Gigapixel photographs of paintings.
Another way are painting applications which simulate brush strokes virtually.
But could these two approaches be combined such that a painting can be made up of digital brush strokes that replicate the original brush strokes in a painting?

Such a combined approach would help to promote the current development in this area and bring interesting applications in other fields.
Two exemplary applications would be the conservation of images and the study of paintings.

As far as conservation goes, an image that is disassembled into its individual brush strokes will store information about the original painting in a more efficient manner than Gigapixel images or 3D scans.
Both of these archiving methods generate huge amounts of data, which is why only a few hundred images in the world are archived in this way \cite{googleartproject}.
Having extracted brush stroke information at hand could store this information more efficiently and improve reconstruction as simulation techniques improve as well.

Considering the study of paintings, there are already computer assisted methods to distinguish forgeries from real artworks.
A way of extracting brush strokes from paintings would open up new possibilities to study the style and techniques of an artists that goes beyond visual inspection.

At last there are other fields that deal with artworks and could benefit from from such an approach such as artistic style transfer in computer vision.
Artistic style transfer relies exclusively on images of paintings when it comes to learning the style of an artist.
As it has been laid out images do not capture everything that makes up a painting and it would be interesting if style transfer could benefit from more advanced information about a painting.

This thesis will perform an experimental evaluation whether it is possible to extract individual brush strokes from a photograph of a painting.
Furthermore, it shall make a first attempt at performing style transfer using brush strokes.

As there is a vast realm of different painting techniques and materials, this work shall only be evaluated on oil painting brush strokes, as there is little data available on other techniques and it is quite possibly the easiest to replicate as well.
Even though there are many artists which created oil paintings in the past, this work will mainly focus on later works by van Gogh.
These works are well known and thus there are many high quality photographs of his works.
Also, most of these paintings show very clear brush strokes which is decisive of van Gogh's style.


\todo{introduce the appraoch broadly}

\paragraph{Structure of this work}
\todo{reformulate}
First, the theoretical background for such an attempt along with related work will be presented.
Then the approach itself will be inferred in two parts that describe the training of a differentiable renderer
and the subsequent retrieval of brush strokes from an image.
This is followed by experiments as well as an ablation study of the approach.
Lastly, there will be a discussion about the results and the future direction of research.



\paragraph{Allocation of individual contributions}
The results, which are presented in this thesis are the result of the combined efforts by Dmitryo Kotovenko, Matthias Wright and Arthur Heimbrecht.




%Since the arrival of electric-based computers in the 20th century, the range of %application for computers and algorithms has steadily grown.
%From the first simple arithmetic calculations over solving complex equations and simulating
%realistic systems to media production and funny apps for phones; computers seem to
%seize every aspect of everyone's life.
%This includes areas that stood their ground for long time against the automation
%that has become common for so many areas like production, communication \etc.
%Recent advances in Machine Learning and the ever increasing efficiency, availability
%and sheer power of computers has put the thought of professions like lawyers and physicians being
%'future-proof' into question.
%
%This includes even the field of arts.
%As art comes can take pretty much every form there is and as it depends on the viewer,
%it is valid to ask whether art could ever be understood by anything but a human being.
%Nonetheless, people have tried repeatedly and proven that Machines CAN grasp broad
%concepts in art, even the delicate art of drawing
%
%\paragraph{Evolution of Art and Style Transfer}
%Since the late 1980's the field of computer vision has looked not only at photographs
%and videos but also started working on images of artworks.
%From early computer based analysis and identification of images that is used even
%today for identifying counterfeits, the field has gone as far as creating art on
%its own.
%Ever since Leon Gatys introduced his paper on 'Neural Style Transfer' in 2016, the quality
%of computer generated paintings has come closer and closer to being indistinguishable to
%an image of an original artwork.
%%At the same time techniques have continuously improved in order to replicate the style of
%%images, periods, techniques or artists.
%This resulted in many interpretations of given images in the style of well-known
%artists like van Gogh, Matisse or Picasso.
%
%\paragraph{Representation Gap}
%Yet, many of these accomplishments lack the final step towards creating authentic artworks,
%which is the transitions from the image domain to the actual artwork domain.
%As images are mere projections of the artworks they try to represent, they lack 
%some of the original artwork's content and message.
%Most people would agree that it is an entirely different experience to view a painting
%in real life than viewing an image of this painting on a screen or sheet of paper.
%
%For once, paint on canvas has depth to it, that can hardly be visualized by a 2D plane.
%Every viewing angle of an image gives away different details in the works of the artist.
%How much paint did the artist use?
%How many layers of color were applied?
%Which brush or technique did were used?
%All of this plays into an artwork's impression as much as the distribution of color does.
%This is the reason why people are interested in actual painted replicas of their favorite
%artworks, instead of just high resolution prints.
%
%As this stands, it becomes clear why an image will always loose some of this information
%about the artwork.
%Sure, one could always make hundreds of very high resolution images or even scan
%the image in 3D space, yet without advanced technology like VR headsets or the like,
%it still does not quite replace the real deal.
%
%\paragraph{From Pixels to Strokes}
%So why even bother then?
%Even with all the shortcomings which were described here, images still catch most of
%the essence of an artwork -- at least enough to train neural networks on this.
%But is it possible to maybe depict artworks as what they actually are, a composition
%of color particles on canvas?
%It just so happens that there are works such as \cite{adobe stroke paper} that
%introduced the idea of synthesizing brush strokes through fluid dynamics and neural networks.
%Unfortunately this comes with a high computational burden.
%So maybe one could start by at least taking a step towards this sort of representation
%by depicting an artwork through a set of brush strokes.
%Ideally, this could help to better capture the essence of an artwork is this comes
%closer to its real world representation and later even open the door for style transfer
%to be performed on this representation as well.


