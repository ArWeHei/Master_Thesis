Ongoing research in art-related computer vision has led to impressive results in recent years.
However, most state-of-the-art approaches are pixel and filter-based and omit the three-dimensional nature of paintings.
This thesis aims to suggest an approach that extracts brushstroke information from images of paintings.
The approach is based on a differentiable renderer, which allows optimizing brushstroke parameters directly through backpropagation and gradient descent.
Applying the same approach to photographs allows stylization of these photographs as paintings.
The results for the proposed approach are mixed.
Although it falls short of accurately extracting brushstrokes from paintings, it conserves of group dynamics and outperforms existing approaches.
In contrast, stylization is on the same level as state-of-the-art painterly rendering approaches.
