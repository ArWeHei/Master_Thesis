%\setchapterstyle{lines}
%\labpage{appendix}
%\blinddocument
\setchapterpreamble[u]{\margintoc}
\chapter{Appendix}
\section{FluidPaint}
\labsec{fluidpaint}
This section shall explain some of the tricks that were employed to generate a nice-looking data set with \citetitle{fluidpaint}.

Like many 

\section{Vecotrizing Alpha Blending}
\labsec{alphablend}

The easiest way to accomplish this, is to first find the maximum blending-depth over all pixels, where the blending-depth $k$ is the number of layers where the alpha value is not zero.
\begin{align}
    k = \argmax_{p \in \text{pixels}} \sum_{i \in \# \text{layers}} \mathds{1}(\alpha_i > 0)
\end{align}

Then the top $k$ layer indices for each pixel are picked, which reduces the number of blending operations from $(n-1)$ to $(k-1)$.
Importantly, the top $k$ indices should not be ordered by their alpha values but remain in the order that was imposed by sorting according to the accuracy value.
Otherwise, the order will most certainly be mixed up, and the pixel with the highest alpha value will always lie on top instead of the pixel that belongs to the most accurate brushstroke.
Especially, as brushstroke renderings fade out towards their edges, this makes a significant difference.
Another way of accelerating the process of alpha-blending is \textbf{vectorizing}.
Instead of iteratively applying the computations, a vectorized operation can perform these computations in parallel.
Vectorizing makes it necessary to construct a tensor with the following properties:

For $\tensor{I} \in [0, 1]^{H \times W \times 4} $ the image target, the shape will be defined
as $\mathcal{S}(\tensor{I}) = (H, W, 4)$.
Each alpha channel will have the values $\alpha^{hw} \in [0, 1]$ for $h = 0, ..., H$ and $w = 0, ..., W$.

The set of rendered and padded brushstrokes $\tensor{J}$ will have the shape $(N, H, W, 4)$ with $N$ depicting the number of brushstrokes that ought to be stitched simultaneously.

Now, looking at each individual pixel in $\tensor{J}$, which is described by $(z^{hw}_n, \alpha^{hw}_n)$ for $n = 1, ..., N$ and $z^{hw}_n \in [0, 1]^{3}$, $z^{hw}$ describes the RGB values and $\alpha^{hw}$ the alpha-channel for a pixel at $(h, w)$.

A blending operation can then be defined by
\begin{align}
    z'^{hw} & = \tilde{\alpha}^{hw} \cdot z^{hw} \\
    \text{or} \\
    z'^{hw} & = \sum_{n=1}^N \tilde{\alpha}^{hw}_n  z^{hw}_n \\
\end{align}

with $z'^{hw}$ the resulting RGB values of the blended pixel and $\tilde{\alpha}^{hw}$ a vector that holds the merged alpha values for each pixel:
\begin{align}
    \tilde{\alpha}^{hw} & =
    \begin{pmatrix}
        \alpha^{hw}_1 & &\\
        \alpha^{hw}_2 & (1 - \alpha^{hw}_1) &\\
        \alpha^{hw}_3 & (1 - \alpha^{hw}_2) & (1 - \alpha^{hw}_1)\\
        \vdots & &\\
    \end{pmatrix}
    \\
    & = \alpha^{hw} \odot 
    \begin{pmatrix}
        1  &\\
        (1 - \alpha^{hw}_1) &\\
        (1 - \alpha^{hw}_2) & (1 - \alpha^{hw}_1)\\
        \vdots &\\
    \end{pmatrix}
    \\
    \rightarrow  \tilde{\alpha}^{hw}_n & = \alpha^{hw}_n \prod^{n-1}_{i=1} (1 - \alpha^{hw}_i)
\end{align}

with $\odot$ the element-wise product

What is left, is to find a way to construct $\tilde{\alpha}^{hw}$ from $\alpha^{hw}$.

For this an auxiliary matrix $\beta^{hw}$ is constructed:
\begin{align}
    \beta^{hw} = \alpha^{hw} \times \mathbb{1}_{1 \times N} = 
    \begin{pmatrix}
        \alpha^{hw}_1 & \alpha^{hw}_2 & \hdots & \alpha^{hw}_n\\
        \alpha^{hw}_1 & \alpha^{hw}_2 & \hdots & \alpha^{hw}_n\\
        \vdots & \vdots & \ddots & \vdots \\
        \alpha^{hw}_1 & \alpha^{hw}_2 & \hdots & \alpha^{hw}_n\\
    \end{pmatrix}
\end{align}
with
$$
\mathbb{1}_{1 \times N} = \begin{pmatrix}
        1 \\
        1 \\
        \vdots\\
        1 \\
    \end{pmatrix}^T
$$

Then $\beta^{hw}$ is strictly triangulated such that:
\begin{align}
    \gamma^{hw} & = \beta^{hw} \odot
    \begin{pmatrix}
        0 & 0 & 0 &\hdots & 0\\
        1 & 0 & 0 &\hdots & 0\\
        1 & 1 & 0 &\hdots & 0\\
        \vdots & \vdots & \ddots & \ddots & \vdots \\
        1 & 1 & \hdots & 1 & 0\\
    \end{pmatrix}
    \\ & = 
    \begin{pmatrix}
        0 & 0 & 0 & \hdots & 0\\
        \alpha^{hw}_1 & 0 & 0 & \hdots & 0\\
        \alpha^{hw}_1 & \alpha^{hw}_2 & 0 & \hdots & 0\\
        \vdots & \vdots & \ddots & \ddots & \vdots \\
        \alpha^{hw}_1 & \alpha^{hw}_2 & \hdots & \alpha^{hw}_{n-1} & 0\\
    \end{pmatrix}
    \\
    \rightarrow \delta^{hw} = 1 - \gamma^{hw} & = 
    \begin{pmatrix}
        1 & 1 & 1 & \hdots & 1\\
        (1 - \alpha^{hw}_1) & 1 & 1 & \hdots & 1\\
        (1 - \alpha^{hw}_1) & (1 - \alpha^{hw}_2) & 1 & \hdots & 1\\
        \vdots & \vdots & \ddots & \ddots & \vdots \\
        (1 - \alpha^{hw}_1) & (1 - \alpha^{hw}_2) & \hdots & (1 - \alpha^{hw}_{n-1}) & 1\\
    \end{pmatrix}
\end{align}

By multiplying the elements along each row in $\delta^{hw}$ one gets:

\begin{align}
    \epsilon^{hw}_i = \prod^N_{j=1} \delta^{hw}_{ij} & =
    \begin{pmatrix}
        1  &\\
        (1 - \alpha^{hw}_1) &\\
        \vdots &\\
        \prod^{N-1}_{j=1} (1 - \alpha^{hw}_j)
    \end{pmatrix}
    \\ \rightarrow \tilde{\alpha}^{hw} = \epsilon^{hw} \odot \alpha^{hw} & =
    \begin{pmatrix}
        \alpha^{hw}_1 & &\\
        \alpha^{hw}_2 & (1 - \alpha^{hw}_1) &\\
        \alpha^{hw}_3 & (1 - \alpha^{hw}_2) & (1 - \alpha^{hw}_1)\\
        & \vdots &\\
        \alpha^{hw}_N & \prod^{N-1}_{j=1} (1 - \alpha^{hw}_j)
    \end{pmatrix}
\end{align}

This vectorized version of alpha blending will introduce a new possible bottleneck as it is, since $\beta^{hw}$ will be a tensor of shape $(N, N, H, W)$, which will equate to
$$
256 \times 256 \times 256 \times 256 \times 4 \si{\byte} = 2^{36}\si{\byte} = 64\si{\gibi\byte}
$$
alone.

This is where the previous position-aware alpha blending tricks becomes useful.
By computing $\beta^{hw}$ only through the top $k$ values of $\alpha^{hw}$ instead of the full tensor $\alpha^{hw}$, the size will be reduced to
$$
k \times k \times 256 \times 256 \times 4 \si{\byte} = k^2 2^{22}\si{\byte} = k^2 \times 4\si{\mebi\byte}
$$
as the shape is reduced to $(k, k, H, W)$.

Ultimately, this accelerates optimization by a factor of $2-3$, as it will be shown in \ref{sec:exp:vectorization}.

It must be mentioned that the upper boundary for computational complexity in using this kind of alpha-blending is $\mathcal{O}(N \log k)$, since the top $k$ search is bound by this complexity.
