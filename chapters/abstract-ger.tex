Entwicklungen im Bereich der Computer Vision mit Bezug zur Kunst habe interesante Ergebnisse in the letzten Jahren hervorgebracht.
Insbesondere Neural Style Transfer~\cite{gatys} und folgende Publikationen haben dazu beigetragen Computer Vision als Feld zu prägen.
Jedoch basieren die meistens Anwendugnen auf Filtern und Pixelrepräsentationen.
Dadurch berücksichtigen sie nicht die dreidimensionale Natur, die Kunst wie Gemälden zugrunde liegt.
In dieser Thesis wird ein Herangehensweise vorgeschlagen, die verspricht Pinselstriche aus Bildern von Gemälden zu extrahieren.
Dafür werden die Bilder durch diesselben Pionselstriche rekonstruiert.
Der Ansatz basiert auf einem differenzierbaren Renderer, welcher es ermöglicht, die Pinselstrichparameter direkt mittels Bakcpropagation und Gradientenverfahren zu optimieren.
Indem derselbe Ansatz auf Fotos angewandt wird, kann auch die Stilisierung von Bildern als Gemälde erreicht werden.
Die Ergebnisse dieser Methode sind von gemischter Natur.
Einerseits kann die gewünschte Genauigkeit beim extrahierung von Pinselstrichen nicht erreicht werden, obwohl Gruppendynamiken erhalten zu bleiben scheinen.
Andererseits erreicht die Stilisierung ein ähnliches Niveau wie die besten Stilisierungsmethoden, die auch auf Pinselstrichen basieren.

